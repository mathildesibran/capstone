% Advanced Programming 2025 - Project Report Template
% HEC Lausanne / UNIL
\documentclass[11pt,a4paper]{article}

% Packages
\usepackage[utf8]{inputenc}
\usepackage[T1]{fontenc}
\usepackage[english]{babel}
\usepackage{amsmath,amssymb,amsthm}
\usepackage{graphicx}
\usepackage{xcolor}
\usepackage{listings}
\usepackage{hyperref}
\usepackage[margin=1in]{geometry}
\usepackage{fancyhdr}
\usepackage{float}
\usepackage{caption}
\usepackage{subcaption}
\usepackage{biblatex}
\addbibresource{references.bib} % optional, if you want a .bib file

% Code listing settings
\definecolor{codegreen}{rgb}{0,0.6,0}
\definecolor{codegray}{rgb}{0.5,0.5,0.5}
\definecolor{codepurple}{rgb}{0.58,0,0.82}
\definecolor{backcolour}{rgb}{0.95,0.95,0.92}

\lstdefinestyle{pythonstyle}{
    backgroundcolor=\color{backcolour},   
    commentstyle=\color{codegreen},
    keywordstyle=\color{magenta},
    numberstyle=\tiny\color{codegray},
    stringstyle=\color{codepurple},
    basicstyle=\ttfamily\footnotesize,
    breakatwhitespace=false,         
    breaklines=true,                 
    captionpos=b,                    
    keepspaces=true,                 
    numbers=left,                    
    numbersep=5pt,                  
    showspaces=false,                
    showstringspaces=false,
    showtabs=false,                  
    tabsize=2,
    language=Python
}

\lstset{style=pythonstyle}

% Header and footer
\pagestyle{fancy}
\fancyhf{}
\rhead{Advanced Programming 2025}
\lhead{Project Report}
\rfoot{Page \thepage}

% Title page information - MODIFY THESE
\title{%
    \Large \textbf{Advanced Programming 2025} \\
    \vspace{0.5cm}
    \LARGE \textbf{Calendar Anomalies and Machine Learning for Daily US Stock Returns} \\
    \vspace{0.3cm}
    \large Final Project Report
}
\author{
    Mathilde Sibran \\  % <-- change if needed
    \texttt{mathilde.sibran@unil.ch} \\  % <-- put your UNIL email
    Student ID: 21682265          % <-- change to your ID
}
\date{\today}

\begin{document}

\maketitle
\thispagestyle{empty}

\begin{abstract}
\noindent
This project studies whether well-known calendar anomalies in financial markets
can be exploited to predict the \emph{direction} of daily stock returns using
classical machine learning models. Using a panel of S\&P~500 constituents from
2010 to 2025, I first construct a clean dataset of 40 liquid US stocks with
limited missing data and compute daily returns and excess returns over the
S\&P~500 index. I engineer calendar-based features capturing the day-of-week
effect, the January effect, turn-of-the-month, holiday-related dummies and
quarter boundaries, together with simple technical indicators (20-day moving
average and volatility).

I then perform a descriptive anomaly analysis to quantify excess returns on
anomaly days globally, by ticker and by sector. Finally, I train four
supervised models --- logistic regression, random forest, gradient boosting and
a neural network --- to classify next-day excess returns as positive or
negative, using a temporal train/test split and SMOTE to handle mild class
imbalance. Results show very small but statistically detectable anomaly
premia; however, all models achieve test accuracies close to 51\% and do not
beat a naive majority-class baseline. The main contribution of the project is
a fully reproducible Python pipeline that links classical market anomaly
literature with modern machine learning practice and highlights the
difficulty of extracting economically meaningful predictability from
daily stock returns.
\end{abstract}

\vspace{0.5cm}
\noindent\textbf{Keywords:} calendar anomalies, stock returns, S\&P~500, machine learning, Python

\newpage
\tableofcontents
\newpage

% ================== MAIN CONTENT ==================

\section{Introduction}
\label{sec:introduction}

Financial markets are often modelled as approximately efficient, which implies
that simple trading rules should not systematically beat a passive benchmark.
However, a long line of empirical work has documented so-called
\emph{calendar anomalies}: patterns in asset returns linked to the day of the
week, the month of the year or specific holidays. Famous examples include the
day-of-the-week effect, the January effect or the pre-holiday effect. These
patterns challenge the strong form of the Efficient Market Hypothesis and are
frequently discussed in both academic papers and practitioner reports.

The research question of this project is the following:

\begin{quote}
\textbf{Can calendar anomalies and simple technical indicators be used, together
with standard machine learning models, to predict the sign of daily excess
returns for US stocks?}
\end{quote}

This matters for at least two reasons. From an academic perspective, it helps
to assess whether anomaly premia are strong and stable enough to survive
out-of-sample testing and realistic modelling choices. From a practical
perspective, even a small but robust edge in directional prediction could
justify the use of such signals in trading strategies, after accounting for
transaction costs and risk.

The objectives of the project are:
\begin{itemize}
    \item to build a clean, well-documented dataset of daily prices for a
    subset of S\&P~500 stocks between 2010 and 2025 and to compute returns and
    excess returns over the market;
    \item to encode several calendar anomalies as dummy features and analyse
    their impact on average excess returns globally, by ticker and by sector;
    \item to train and evaluate several machine learning classifiers on the
    binary prediction task ``next-day excess return positive vs.\ negative'';
    \item to provide a clear, reproducible Python implementation with a single
    entry point script \texttt{main.py}.
\end{itemize}

The report is organised as follows. Section~\ref{sec:literature} briefly
summarises related work on calendar anomalies and machine learning in finance.
Section~\ref{sec:methodology} describes the data, features, modelling
approach and implementation. Section~\ref{sec:results} presents the empirical
results. Section~\ref{sec:discussion} discusses their interpretation and
limitations. Section~\ref{sec:conclusion} concludes and outlines directions
for future work.

\section{Literature Review / Related Work}
\label{sec:literature}

Calendar anomalies have been documented for decades. The day-of-the-week
effect refers to systematic differences between returns on different weekdays,
often with lower returns on Mondays and higher returns on Fridays. The January
effect describes the tendency for stocks, in particular small caps, to earn
higher returns in January than in other months. Holiday effects capture
unusually positive returns on the days preceding market holidays. These
findings appear in many developed markets and asset classes, although the
magnitude and persistence of the anomalies vary across samples and periods.

At the same time, advances in machine learning have led to increasing interest
in applying classification and regression models to financial time series.
Typical algorithms include logistic regression, tree-based ensembles such as
random forests and gradient boosting, and shallow neural networks. In most
studies, daily returns are extremely noisy and close to unpredictable after
transaction costs; models tend to deliver modest improvements over random
guessing at best.

This project sits at the intersection of these two strands. Rather than
attempting to predict raw returns purely from technical indicators, I focus on
a small set of interpretable calendar-based features and ask whether they
carry enough information to help a range of standard machine learning models
beat a naive baseline on out-of-sample data.

\section{Methodology}
\label{sec:methodology}

\subsection{Data Description}

The main dataset is an Excel file \texttt{market\_anomalie.xlsx} supplied as
part of the course. It contains two sheets:

\begin{itemize}
    \item \textbf{DAILY}: daily closing prices for 490 tickers that are (or
    have been) part of the S\&P~500 index, over the period from 2010-01-01 to
    2025-11-01 (4\,141 trading days). The rows correspond to dates and the
    columns to tickers.
    \item \textbf{MONTHLY}: a lower-frequency version with monthly prices,
    which is not directly used in the modelling but could be useful for
    robustness checks.
\end{itemize}

Because not all tickers have complete histories, I first compute the
percentage of missing daily prices for each ticker. I keep only those with
less than 5\% missing data and then select the 40 stocks with the highest
return volatility, measured by the standard deviation of daily log-returns.
This provides a liquid and informative subset while keeping the computation
manageable.

In addition, daily market index data are downloaded from \texttt{yfinance}:

\begin{itemize}
    \item S\&P~500 index (\texttt{\^GSPC}), used as the market benchmark;
    \item STOXX Europe 600 index (\texttt{\^STOXX}), downloaded but not used
    directly in the current version of the project.
\end{itemize}

The S\&P~500 close prices are aligned on the same calendar as the stock
prices. After reshaping the wide price matrix into long format
(\texttt{Date}, \texttt{ticker}, \texttt{close\_price}) and merging with the
index data, the final dataset contains about 165\,000 observations.

A separate CSV file \texttt{sector\_mapping.csv} provides sector labels for a
subset of the 40 tickers. In practice, only the Energy sector is fully
represented in this mapping, which limits the scope of the sector-level
analysis.

\subsection{Approach}

\subsubsection*{Target and basic returns}

For each stock-day observation, I compute:

\begin{itemize}
    \item the stock daily simple return
    \[
        r_{i,t} = \frac{P_{i,t} - P_{i,t-1}}{P_{i,t-1}} ;
    \]
    \item the market return \(r^{M}_{t}\) from S\&P~500 index closes;
    \item the excess return \(er_{i,t} = r_{i,t} - r^{M}_{t}\).
\end{itemize}

The binary target variable \(y_{i,t}\) for classification is
\[
    y_{i,t} =
    \begin{cases}
        1 & \text{if } er_{i,t} > 0,\\
        0 & \text{otherwise.}
    \end{cases}
\]

\subsubsection*{Calendar anomaly features}

The following calendar-based dummy variables are constructed:

\begin{itemize}
    \item \texttt{day\_of\_week}: encoded as an integer 0--4 (Monday--Friday) and optionally one-hot encoded in the models;
    \item \texttt{month}: month of the year (1--12);
    \item \texttt{is\_january}: indicator for January (January effect);
    \item \texttt{turn\_of\_month}: indicator equal to 1 for days around the turn of the month;
    \item \texttt{pre\_holiday}: indicator for trading days immediately preceding an exchange holiday;
    \item \texttt{is\_christmas}, \texttt{is\_new\_year}, \texttt{is\_thanksgiving};
    \item \texttt{is\_first\_day\_quarter}: first trading day of each calendar quarter;
    \item \texttt{sell\_in\_may}: indicator related to the ``Sell in May and go away'' effect.
\end{itemize}

These variables allow me to reproduce standard calendar anomalies and to use
them jointly as predictors in the classification models.

\subsubsection*{Technical indicators}

To complement the calendar features, I compute simple technical indicators
based on each stock's own price history:

\begin{itemize}
    \item \texttt{ma20}: 20-day moving average of the stock return;
    \item \texttt{volatility\_20d}: 20-day rolling standard deviation of returns;
    \item \texttt{daily\_return}: same as \(r_{i,t}\);
    \item \texttt{market\_return}: daily S\&P~500 return.
\end{itemize}

These indicators capture short-term momentum and volatility and are common in
quantitative trading strategies.

\subsubsection*{Train/test split and resampling}

To avoid look-ahead bias, I perform a temporal split:

\begin{itemize}
    \item \textbf{Train period:} 2010-01-29 to 2018-12-31 (92\,769 observations);
    \item \textbf{Test period:} 2019-01-01 to 2025-11-14 (71\,760 observations).
\end{itemize}

The target distribution is slightly imbalanced, with around 48\% of positive
excess returns in the raw training data. I therefore apply SMOTE
(Synthetic Minority Over-sampling Technique) on the training set only, which
increases the number of minority-class samples and brings the class proportion
closer to 50/50. The test set is left untouched.

\subsubsection*{Models and evaluation metrics}

I train four models implemented in scikit-learn:

\begin{itemize}
    \item \textbf{Logistic regression} with \(\ell_{2}\) regularisation;
    \item \textbf{Random forest} classifier;
    \item \textbf{Gradient boosting} classifier, with hyperparameters tuned by
    grid search (best found: learning rate 0.05, max depth 2, 50 estimators);
    \item \textbf{Multilayer perceptron} classifier (shallow neural network).
\end{itemize}

As benchmark, I use the accuracy of a trivial classifier that always predicts
the majority class in the training set.

Performance is evaluated primarily by:

\begin{itemize}
    \item \textbf{Accuracy}: overall proportion of correct predictions;
    \item \textbf{AUC}: area under the ROC curve, using predicted probabilities;
    \item \textbf{Confusion matrix} and \textbf{precision/recall/f1-score} by class.
\end{itemize}

Given the financial context, even small improvements over 50\% can in theory
be interesting, but they must be consistent and robust.

\subsection{Implementation}

The entire project is implemented in Python and organised as a small package
under the \texttt{src/} directory:

\begin{itemize}
    \item \texttt{data\_loader.py}: reading Excel data, downloading indices
    from \texttt{yfinance}, selecting the top 40 stocks and structuring the
    long-format dataset with market returns and sector labels;
    \item \texttt{features.py}: computation of daily and excess returns,
    calendar anomaly dummies and rolling technical indicators;
    \item \texttt{anomalies.py}: descriptive analysis of day-of-week and
    January effects, with results exported to Excel tables;
    \item \texttt{anomaly\_analysis.py}: unified analysis of anomaly effects
    on excess returns (global, by ticker, by sector) and generation of
    summary plots;
    \item \texttt{models.py}: definitions of the four machine learning models,
    the temporal train/test split, SMOTE resampling and evaluation functions;
    \item \texttt{main.py}: entry point that orchestrates the full pipeline.
\end{itemize}

The main script can be summarised as follows:

\begin{lstlisting}[caption={Main pipeline in \texttt{main.py}}]
def main():
    print("🚀 Starting full pipeline...")

    # 1. Load Excel and sector mapping
    excel_data = load_excel_data("data/raw/market_anomalie.xlsx")
    sector_mapping = load_sector_mapping()

    # 2. Download market indices
    market_indices = download_market_indices()

    # 3. Select top stocks
    selected_tickers = select_top_stocks(excel_data["sp500_daily"])

    # 4. Clean and structure long-format dataset
    df = clean_and_structure_data(
        excel_data["sp500_daily"],
        selected_tickers,
        market_indices["sp500_index"],
        sector_mapping
    )

    # 5. Feature engineering
    df = create_features(df)

    # 6. Descriptive anomaly analysis
    analyze_weekday_effect(df)
    analyze_january_effect(df)
    run_anomaly_return_analysis(df)

    # 7. Machine learning models
    run_logistic_regression(df)
    run_random_forest(df)
    run_gradient_boosting(df)
    run_neural_network(df)

    print("🎉 Pipeline COMPLET terminé !")
\end{lstlisting}

The code runs on the Nuvolos environment used in the course (1~vCPU, 4~GB
RAM, Python~3.12) and relies mainly on \texttt{pandas}, \texttt{numpy},
\texttt{scikit-learn}, \texttt{imbalanced-learn}, \texttt{matplotlib},
\texttt{seaborn} and \texttt{yfinance}.

\section{Results}
\label{sec:results}

\subsection{Experimental Setup}

All experiments follow the temporal split described above. Feature engineering
is applied identically to train and test sets, with any standardisation or
encoding steps fitted on the training period only and applied to the test
period.

The main hyperparameters are:
\begin{itemize}
    \item Logistic regression: default scikit-learn settings with
    \(\ell_{2}\) penalty;
    \item Random forest: default settings with 100 trees;
    \item Gradient boosting: learning rate 0.05, max depth 2, 50 estimators;
    \item Neural network: one hidden layer with ReLU activation and
    early stopping enabled (default scikit-learn configuration).
\end{itemize}

SMOTE is applied only on the training set. Test performance is always reported
on the original, untouched test data.

\subsection{Performance Evaluation}

Table~\ref{tab:performance} summarises the test accuracy and AUC of all
models, together with the majority-class baseline. Values are rounded from the
logged outputs of the pipeline.

\begin{table}[H]
\centering
\caption{Model performance on the test period (2019--2025)}
\label{tab:performance}
\begin{tabular}{|l|c|c|}
\hline
\textbf{Model} & \textbf{Accuracy} & \textbf{AUC} \\
\hline
Baseline (majority class) & 0.520 & -- \\
Logistic Regression       & 0.515 & 0.514 \\
Random Forest             & 0.507 & 0.515 \\
Gradient Boosting         & 0.507 & 0.520 \\
Neural Network (MLP)      & 0.503 & 0.517 \\
\hline
\end{tabular}
\end{table}

The confusion matrices and classification reports produced by the code show
roughly symmetric precision and recall for both classes, with macro-averaged
precision, recall and f1-scores all close to 0.51. The SMOTE resampling avoids
severe bias towards the majority class, but the overall predictive power
remains low.

\subsection{Anomaly Return Analysis and Visualizations}

The module \texttt{anomaly\_analysis.py} produces several Excel tables:

\begin{itemize}
    \item \texttt{anomaly\_global\_returns.xlsx}: global statistics for each
    anomaly (mean excess return on anomaly vs.\ non-anomaly days, difference
    and t-statistic);
    \item \texttt{anomaly\_ticker\_returns.xlsx}: for each ticker and anomaly,
    mean excess return on anomaly days;
    \item \texttt{anomaly\_sector\_returns.xlsx}: anomaly statistics by sector
    (for the subset of stocks with sector labels).
\end{itemize}

Figure~\ref{fig:global_diff} shows the bar chart of average excess return
differences between anomaly and normal days. While some anomalies such as
day-of-week and January display positive differences, magnitudes are small in
absolute terms.

\begin{figure}[H]
\centering
\includegraphics[width=0.8\textwidth]{results/anomalies/plots/global_anomaly_diff.png}
\caption{Average excess return difference (anomaly vs.\ non-anomaly days) by anomaly.}
\label{fig:global_diff}
\end{figure}

For the sector analysis, the limited coverage of sector labels means that only
the Energy sector is reliably represented. Its average daily return is
slightly positive over the sample, with volatility around 3\%. The corresponding
heatmap of anomaly effects by sector and anomaly is shown in
Figure~\ref{fig:sector_heatmap}.

\begin{figure}[H]
\centering
\includegraphics[width=0.8\textwidth]{results/anomalies/plots/sector_anomaly_heatmap.png}
\caption{Heatmap of excess return differences by sector and anomaly (where available).}
\label{fig:sector_heatmap}
\end{figure}

Variable importance plots from the random forest and gradient boosting models
confirm that features such as \texttt{ma20}, \texttt{volatility\_20d},
\texttt{market\_return}, \texttt{month} and \texttt{day\_of\_week} contribute
most to the predictions, while some holiday dummies have negligible impact.

\section{Discussion}
\label{sec:discussion}

\subsection*{Why do models not beat the baseline?}

Despite the careful construction of features and the use of several machine
learning algorithms, none of the models significantly outperforms the naive
majority-class classifier. There are several reasons for this:

\begin{itemize}
    \item Daily excess returns are extremely noisy and have very low signal-to-noise ratio. Most of the variation is essentially random from the
    perspective of our simple feature set.
    \item Calendar anomalies, even when statistically significant in mean
    returns, tend to have very small magnitudes (basis points per day). Such
    small effects are difficult to exploit in a high-variance classification
    setting.
    \item The models are intentionally kept simple and conservative to avoid
    overfitting. More complex architectures might fit the training data better
    but would likely not generalise.
    \item Features are based only on past prices and calendar information; no
    fundamental or cross-sectional information is used.
\end{itemize}

\subsection*{What was surprising?}

Two aspects were somewhat surprising:

\begin{itemize}
    \item The descriptive anomaly analysis did show small but visible
    differences in excess returns on some anomaly days, especially for some
    tickers and for the Energy sector. However, these differences did not
    translate into strong predictive performance for individual stock returns.
    \item Gradient boosting and the neural network achieved slightly higher
    AUC than accuracy, suggesting that the models capture some marginal
    ranking information about the probability of positive excess returns, even
    if the final classification at threshold 0.5 remains weak.
\end{itemize}

\subsection*{Limitations}

The project has several limitations:

\begin{itemize}
    \item The sector mapping is incomplete, which restricts the sector-level
    analysis to a small subset of stocks.
    \item Transaction costs, bid-ask spreads and other frictions are ignored.
    Even if the models slightly beat 50\% accuracy, such small edges would not
    survive realistic trading costs.
    \item Hyperparameter tuning, especially for the neural network, is fairly
    limited due to time and computational constraints.
    \item Only one train/test split is considered; a rolling or expanding
    window evaluation could provide a more robust assessment of stability over
    time.
\end{itemize}

\section{Conclusion and Future Work}
\label{sec:conclusion}

\subsection{Summary}

This project built a complete Python pipeline to investigate whether calendar
anomalies and simple technical indicators can help predict the sign of daily
excess returns for a subset of S\&P~500 stocks. Using data from 2010 to 2025,
I engineered a set of interpretable features, performed global, ticker-level
and sector-level anomaly analyses, and trained four standard machine learning
models.

The main findings are that:

\begin{itemize}
    \item Several calendar anomalies exhibit small but measurable effects on
    average excess returns over the sample;
    \item However, when used as predictors in daily classification models,
    these features yield test accuracies around 51\%, not significantly above
    a majority-class baseline;
    \item Tree-based models highlight the relative importance of moving
    averages, volatility and some calendar variables, but overall predictive
    power remains weak.
\end{itemize}

\subsection{Future Directions}

There are several natural extensions:

\begin{itemize}
    \item Incorporating additional information, such as cross-sectional factors,
    market volatility indices or macroeconomic announcements, could enrich the
    feature set.
    \item Exploring different prediction horizons (e.g.\ weekly instead of
    daily) or portfolio-level targets might yield stronger and more exploitable
    signals.
    \item Performing a rolling-window backtest with transaction costs would
    allow a more realistic evaluation from an investment perspective.
    \item Completing and refining the sector mapping would enable a deeper
    cross-sector comparison of anomaly effects.
\end{itemize}

Overall, the project confirms the intuitive idea that calendar anomalies on
their own are not a ``free lunch'', but it provides a clear and modular code
base that could be reused for more advanced studies.

% ================== REFERENCES ==================
\newpage
\section*{References}
\addcontentsline{toc}{section}{References}

% If using biblatex (recommended)
% \printbibliography[heading=none]

% Or manually:
\begin{enumerate}
    \item Dataset provided in Advanced Programming 2025 course: daily prices
    for S\&P~500 constituents, HEC Lausanne / UNIL.
    \item Pedregosa, F. et al. (2011). \textit{Scikit-learn: Machine Learning in
    Python}. Journal of Machine Learning Research, 12, 2825--2830.
    \item Chawla, N. V., Bowyer, K. W., Hall, L. O., \& Kegelmeyer, W. P. (2002).
    \textit{SMOTE: Synthetic Minority Over-sampling Technique}. Journal of
    Artificial Intelligence Research, 16, 321--357.
\end{enumerate}

% ================== APPENDICES ==================
\newpage
\appendix
\section{Additional Figures}
\label{app:figures}

Additional plots (e.g.\ feature importance bar charts, confusion matrices and
learning curves) can be included here if needed.

\section{Code Repository}
\label{app:code}

\noindent
\textbf{GitHub Repository:} \url{https://github.com/yourusername/project-repo}

\noindent
To reproduce the results:
\begin{itemize}
    \item clone the repository and create the Python environment from
    \texttt{environment.yml};
    \item place the raw data files under \texttt{data/raw/};
    \item run \texttt{python main.py} from the project root.
\end{itemize}

\end{document}
