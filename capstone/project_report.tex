% Advanced Programming 2025 - Project Report
% HEC Lausanne / UNIL

\documentclass[11pt,a4paper]{article}

% Packages
\usepackage[utf8]{inputenc}
\usepackage[T1]{fontenc}
\usepackage[french]{babel}
\usepackage{amsmath,amssymb}
\usepackage{graphicx}
\usepackage{xcolor}
\usepackage{listings}
\usepackage{hyperref}
\usepackage[margin=1in]{geometry}
\usepackage{fancyhdr}
\usepackage{float}
\usepackage{caption}
\usepackage{subcaption}

% Header and footer
\pagestyle{fancy}
\fancyhf{}
\rhead{Advanced Programming 2025}
\lhead{Final Project Report}
\rfoot{Page \thepage}

% Title page
\title{
    \Large \textbf{Advanced Programming 2025} \\
    \vspace{0.5cm}
    \LARGE \textbf{Anomalies calendaires et machine learning pour la prédiction des rendements boursiers journaliers} \\
    \vspace{0.3cm}
    \large Final Project Report
}
\author{
    Mathilde Sibran \\
    \texttt{mathilde.sibran@unil.ch} \\
    Student ID: 21682265
}
\date{\today}

\begin{document}
\maketitle
\thispagestyle{empty}

% ================= ABSTRACT =================

\begin{abstract}
\noindent
Ce projet étudie si les anomalies calendaires bien connues des marchés financiers
peuvent être exploitées pour prédire la direction des rendements excédentaires
journaliers des actions à l’aide de modèles de machine learning. À partir de données
de prix quotidiens pour un sous-ensemble d’actions du S\&P~500 sur la période 2010–2025,
un jeu de données propre de 40 actions liquides est constitué. Les rendements des
actions et du marché sont calculés, et de nombreuses variables calendaires et
techniques sont construites.

Une analyse descriptive est d’abord menée afin de mesurer l’impact des anomalies,
à la fois globalement, par action et par secteur. Plusieurs modèles supervisés sont
ensuite entraînés — régression logistique, random forest, gradient boosting, XGBoost
et réseau de neurones — en utilisant une séparation temporelle apprentissage/test.

Les résultats montrent que, bien que la plupart des modèles présentent un pouvoir
prédictif limité, le modèle Random Forest surperforme nettement les autres avec une
accuracy supérieure à 62\% et un ROC AUC proche de 0.67. Ces résultats suggèrent que
les modèles non linéaires peuvent partiellement exploiter de faibles signaux
liés aux anomalies calendaires, même si la prédictibilité globale reste modérée.
\end{abstract}

\newpage
\tableofcontents
\newpage

% ================= INTRODUCTION =================

\section{Introduction}

Les marchés financiers sont généralement considérés comme relativement efficients,
ce qui implique que les stratégies de trading simples ne devraient pas permettre
de battre durablement le marché. Pourtant, de nombreuses études empiriques ont mis
en évidence l’existence d’anomalies calendaires, telles que l’effet jour de la
semaine, l’effet janvier ou encore l’effet pré-férié.

La question centrale de ce projet est la suivante :

\begin{quote}
Les anomalies calendaires et des indicateurs techniques simples peuvent-ils être utilisés
avec le machine learning pour prédire le signe du rendement excédentaire journalier
des actions américaines ?
\end{quote}

Les objectifs du projet sont :
\begin{itemize}
    \item construire un jeu de données propre de prix quotidiens ;
    \item analyser l’impact des anomalies calendaires ;
    \item entraîner et comparer plusieurs modèles de machine learning ;
    \item proposer une chaîne de traitement entièrement reproductible en Python.
\end{itemize}

% ================= LITTERATURE =================

\section{Revue de littérature}

Les anomalies calendaires telles que l’effet lundi, l’effet janvier ou l’effet
pré-férié ont été largement étudiées en finance empirique. Bien que leur significativité
statistique soit souvent confirmée, leur ampleur économique reste en général très
limitée.

Parallèlement, de nombreuses études ont appliqué des méthodes de machine learning
à la prévision des rendements boursiers. Toutefois, la majorité des résultats montre
que les rendements journaliers sont extrêmement difficiles à prédire de manière fiable.

Ce travail se situe à l’intersection de ces deux approches, en combinant anomalies
calendaires et modèles supervisés de machine learning.

% ================= DATA =================

\section{Description des données}

Le jeu de données est composé de prix de clôture journaliers d’actions américaines
sur la période 2010–2025. Après nettoyage, un sous-ensemble final de 40 actions est
sélectionné sur la base de critères de disponibilité des données et de liquidité.

Les rendements du marché sont mesurés à partir de l’indice S\&P~500. Un fichier de
correspondance permet d’assigner chaque action à son secteur économique. Le jeu de
données final contient environ 160 000 observations journalières.

% ================= FEATURES =================

\section{Variables explicatives et variable cible}

Pour chaque observation action–jour :
\begin{itemize}
    \item rendement journalier de l’action,
    \item rendement du marché,
    \item rendement excédentaire.
\end{itemize}

La variable cible binaire vaut 1 si le rendement excédentaire du lendemain est positif,
et 0 sinon.

Variables calendaires :
\begin{itemize}
    \item jour de la semaine,
    \item mois,
    \item fin de mois,
    \item Sell-in-May,
    \item pré-férié,
    \item Noël, Nouvel An, Thanksgiving,
    \item premier jour de chaque trimestre.
\end{itemize}

Indicateurs techniques :
\begin{itemize}
    \item momentum à 5, 10 et 20 jours,
    \item volatilité glissante à 10 et 20 jours,
    \item moyennes mobiles à 20 et 50 jours,
    \item prix relatif à la moyenne mobile.
\end{itemize}

Des variables muettes de secteur sont ajoutées.

% ================= MODELS =================

\section{Modèles et méthodologie}

Séparation temporelle :
\begin{itemize}
    \item Entraînement : 2010–2018
    \item Test : 2019–2025
\end{itemize}

Modèles utilisés :
\begin{itemize}
    \item Régression logistique
    \item Random Forest
    \item Gradient Boosting
    \item XGBoost
    \item Réseau de neurones (MLP)
\end{itemize}

Métriques :
\begin{itemize}
    \item Accuracy
    \item ROC AUC
\end{itemize}

% ================= IMPLEMENTATION =================

\section{Implémentation}

Le projet est entièrement codé en Python selon une structure modulaire.

\subsection{Architecture}
\begin{itemize}
    \item \texttt{main.py}
    \item \texttt{data\_loader.py}
    \item \texttt{features.py}
    \item \texttt{anomalies.py}
    \item \texttt{models.py}
\end{itemize}

\subsection{Prétraitement}
Chargement des données, calcul des rendements, intégration du secteur.

\subsection{Variables}
Anomalies calendaires, momentum, volatilité, moyennes mobiles, secteurs.

\subsection{Machine learning}
Standardisation via \texttt{Pipeline}, entraînement automatisé, évaluation Accuracy \& AUC.

\subsection{Automatisation}
Un simple \texttt{python main.py} exécute l’ensemble du pipeline.

% ================= CODEBASE =================

\section{Codebase et reproductibilité}

Le projet est entièrement reproductible via un dépôt Git.

Structure :
\begin{itemize}
    \item \texttt{main.py}
    \item \texttt{src/}
    \item \texttt{data/}
    \item \texttt{results/}
    \item \texttt{environment.yml}
    \item \texttt{README.md}
\end{itemize}

Dépendances :
\begin{itemize}
    \item pandas, numpy
    \item scikit-learn
    \item xgboost
\end{itemize}

Commande unique :
\begin{center}
\texttt{python main.py}
\end{center}

% ================= RESULTS =================

\section{Résultats}

\subsection{Performance des modèles}

\begin{table}[H]
\centering
\begin{tabular}{|l|c|c|}
\hline
Modèle & Accuracy & ROC AUC \\
\hline
Régression Logistique & 0.513 & 0.504 \\
Random Forest & \textbf{0.624} & \textbf{0.670} \\
Gradient Boosting & 0.533 & 0.546 \\
XGBoost & 0.548 & 0.568 \\
Réseau de neurones & 0.537 & 0.553 \\
\hline
\end{tabular}
\caption{Performance finale des modèles}
\end{table}

Le modèle Random Forest surperforme nettement les autres.

% ================= LIMITES =================

\section{Limites}

Les rendements journaliers sont extrêmement bruités. Les données fondamentales,
macroéconomiques et de sentiment ne sont pas intégrées.

% ================= FUTURE WORK =================

\section{Perspectives}

Ajout de données intrajournalières, variables macroéconomiques, analyse de portefeuilles,
modèles LSTM.

% ================= CONCLUSION =================

\section{Conclusion}

Les anomalies calendaires ne permettent pas à elles seules de prédire efficacement
les rendements journaliers. Le Random Forest extrait néanmoins un signal partiel.

\end{document}
