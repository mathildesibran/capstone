\documentclass[12pt]{article} %hidelinks


\usepackage{hyperref}
\usepackage{natbib}
\hypersetup{colorlinks,linkcolor={red},citecolor={blue},urlcolor={red}} 
\usepackage[margin=0.7in]{geometry}
\usepackage{graphicx}
\usepackage{grffile}
\usepackage{pdfpages}
\usepackage{amssymb}
\usepackage{amsmath}
\usepackage{graphicx}
\usepackage{subcaption}
\usepackage[font=footnotesize,labelfont=bf]{caption}
\usepackage[font=footnotesize,labelfont=bf]{subcaption}
\usepackage{color}
\usepackage[]{algorithm2e}
\usepackage[utf8]{inputenc}
\usepackage[english]{babel}
\usepackage[toc,page]{appendix}
\usepackage{tikz}
\usetikzlibrary{arrows,backgrounds,calc,trees}
\tikzset{multiline/.style={align=center,anchor=north}}
\usepackage{ifthen}
\usetikzlibrary{positioning}
\usepackage{siunitx}
\usepackage{datatool}
\usepackage{booktabs}
\usepackage{rotating}
\usepackage{bm}
\usepackage{tabularx}

\pgfdeclarelayer{background}
\pgfsetlayers{background,main}


\newcommand{\convexpath}[2]{
[   
    create hullnodes/.code={
        \global\edef\namelist{#1}
        \foreach [count=\counter] \nodename in \namelist {
            \global\edef\numberofnodes{\counter}
            \node at (\nodename) [draw=none,name=hullnode\counter] {};
        }
        \node at (hullnode\numberofnodes) [name=hullnode0,draw=none] {};
        \pgfmathtruncatemacro\lastnumber{\numberofnodes+1}
        \node at (hullnode1) [name=hullnode\lastnumber,draw=none] {};
    },
    create hullnodes
]
($(hullnode1)!#2!-90:(hullnode0)$)
\foreach [
    evaluate=\currentnode as \previousnode using \currentnode-1,
    evaluate=\currentnode as \nextnode using \currentnode+1
    ] \currentnode in {1,...,\numberofnodes} {
  let
    \p1 = ($(hullnode\currentnode)!#2!-90:(hullnode\previousnode)$),
    \p2 = ($(hullnode\currentnode)!#2!90:(hullnode\nextnode)$),
    \p3 = ($(\p1) - (hullnode\currentnode)$),
    \n1 = {atan2(\y3,\x3)},
    \p4 = ($(\p2) - (hullnode\currentnode)$),
    \n2 = {atan2(\y4,\x4)},
    \n{delta} = {-Mod(\n1-\n2,360)}
  in 
    {-- (\p1) arc[start angle=\n1, delta angle=\n{delta}, radius=#2] -- (\p2)}
}
-- cycle
}




\newtheorem{theorem}{Theorem}
\newtheorem{corollary}{Corollary}[theorem]
\newtheorem{lemma}[theorem]{Lemma}
\newtheorem{definition}{Definition}
 

\newcommand{\hui}[1]{\textcolor{red}{\textbf{TBD Hui: #1}}}
\newcommand{\simon}[1]{\textcolor{purple}{\textbf{TBD Simon: #1}}}
\newcommand{\alll}[1]{\textcolor{green}{\textbf{TBD All: #1}}}
\newcommand{\quotes}[1]{``#1''}


\newcommand{\x}{\mathbf{x}}
\newcommand{\g}{\mathbf{g}}
\newcommand{\m}{\mathbf{m}}
\newcommand{\f}{\mathbf{f}}
\newcommand{\nn}{\mathcal{N}_{\bm{\rho}}}
\newcommand{\kb}{\mathbf{k}}
\newcommand{\btheta}{\bm{\theta}}
\newcommand{\dtr}{\mathcal{D}_{\text{train}}}
\newcommand{\birth}{\text{birth}}
\newcommand{\lr}{\alpha_{\text{learn}}}



\newcommand{\R}{\mathbb{R}}
\newcommand{\E}[1]{{\text{E}\left[ #1 \right]}}
\newcommand{\Eind}[2]{{\text{E}_{#1}\left[ #2 \right]}}

\DeclareMathOperator*{\argmax}{arg\,max}
\DeclareMathOperator*{\argmin}{arg\,min}

\newcommand{\citepos}[1]{\citeauthor{#1}'s \citeyear{#1}}
\newcommand{\citeposs}[1]{\citeauthor{#1}' \citeyear{#1}}

\linespread{1.3}


\title{``Introduction to Data Science and Advanced Programming''\\ Requirements for the Capstone Project 2025}
%\author{Simon Scheidegger\footnote{University of Lausanne, email: simon.scheidegger@unil.ch}}
%\date{\today}
\date{Fall term 2025}

\begin{document}

%%%%%%%%%%%%%%%%%%%%%%%%%%%%%%%%%%%%%%%%%%%%%%%%%%%%%%%%%%%%%%%%%%%%%%%%

\maketitle


\section{General Comments and formal requirements}

To show the successful acquisition and mastery of the course content thought in the class
``Advanced Programming'', ALL students individually need to propose and carry out a final project
that applies this knowledge to a problem of their choice.

All students individually will present their project via a 15 minutes long video that has to be submitted/uploaded/sent to the TAs of the course in the last week after classes of the semester---that is, on Sunday, {\bf{15st of December 2025, 23.59h}}.

%Several {\bf{randomly chosen projects and speakers}} will present the 
%project on the last day of the course---that is, on Wednesday, 29th of May.

A pre-recorded presentation of the project is supposed to be 15' long, recorded, for instance, via Zoom.
%{\bf{All the group members}} are expected to talk on the recording. 
Clarification questions before the grading can be raised via email or Zoom call after the submission. 
%{\bf{The presentation will not be part of the grade.}}

Question regarding the details should be directed to Anna Smirnova (anna.smirnova@unil.ch) or the other TAs of the class.


\begin{itemize}

 \item 100\% of the final grade will be based on submitting a project that 
provides evidence of the student's ability to apply what she/he learned 
during class.

\item The grading will be based on several formal factors that are listed 
below as well as on the originality and complexity of the project and the presentation.

%\item We allow a maximum of 3 students to work jointly on a project, 
%however, they need to notify the responsibles of the course (Aryan Eftekhari and Aleksandra Malova).

\item By the {\bf{Monday, November 3rd, 2025}} the latest, all students need to inform the TAs about
the proposed project by submitting a 200-word proposal. 
%and b) about who the team members of the project are. 
Based on the TA's sign-off regarding the scope and complexity
of the proposed project, it then can be undertaken.

\item Due date of the semester project is Sunday, {\bf{21st of December 2025, 23.59h}}.
No late hand-in is accepted.
A late hand-in will result in zero points for the project.

\item The semester project will be submitted to Anna Smirnova (anna.smirnova@unil.ch).


\end{itemize}








%%%%%%%%%%%%%%%%%%%%%%%%%%%%%%%%%%%%%%%%%%%%%%%%%%%%%%%%%%%%%%%%%%%%%%%%

\section{Grading}

%%%%%%%%%%%%%%%%%%%%%%%%%%%%%%%%%%%%%%%%%%%%%%%%%%%%%%%%%%%%%%%%%%%%%%%%


The grading will be based on three factors---that is, 
i) satisfying the formal factors (listed below), as well as on ii) complexity of 
the project and iii) the originality of the project.


\subsection{Formal Factors}

\begin{itemize}
 \item At the due date, three parts have to be submitted. i) a research paper of 10 pages length (min. 8 pages without references), ii) the source code (and auxiliary data if existing) to carry out the programming project, and iii) a recording of about 15 minutes lenght that presents and summarizes the project.
 \item The research paper will be submitted in a pdf format and in the SIAM conference style (style files and examples are provided).
 \item The submitted research paper needs to contain the following mandatory 8 sections
 \begin{itemize}
  \item 1. Abstract
  \item 2. Introduction
  \item 3. Description of the research question and the relevant literature
  \item 4. The methodology/algorithm applied to address the research question 
  and potentially its complexity
 \item 5. A discussion of the implementation of the algorithm (code) and, if possible, its parallel implementation and performance
 \item 6. A description of how to maintain and update the codebase (by using git, unit testing, etc...) 
 \item 7. Results
 \item 8. Conclusion
  \item Appendix: A list of helper-tools (if used, such as Chat-GPT).
 \end{itemize}


 \item Additional sections such as parallel scalability are permitted if they make sense in the project context.
\end{itemize}


You may use Chat-GPT, Co-pilot, and other resources aiding the code development and writing. However, all resources you use need to be clearly listed in the appendix.








% \newpage
% \bibliography{./references}{}
% \bibliographystyle{apalike}



\end{document}
